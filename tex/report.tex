\documentclass[a4paper,12pt,titlepage,finall]{article}

\usepackage[T1,T2A]{fontenc}     % форматы шрифтов
\usepackage[utf8]{inputenc}      % кодировка символов, используемая в данном файле
\usepackage[english, russian]{babel}      % пакет русификации
\usepackage{tikz}                % для создания иллюстраций
\usepackage{pgfplots}            % для вывода графиков функций
\usepackage{geometry}		     % для настройки размера полей
\usepackage{indentfirst}         % для отступа в первом абзаце секции
\usepackage{amsmath,amsthm,amssymb}
\usepackage{mathtext}
\usepackage{graphicx}
\usepackage{hyperref}
\graphicspath{ {./img} }

%Настройка листингов для языка C
\usepackage{xcolor}
\usepackage{listings}
\lstset{extendedchars=\true}

\definecolor{mGreen}{rgb}{0,0.6,0}
\definecolor{mGray}{rgb}{0.5,0.5,0.5}
\definecolor{mPurple}{rgb}{0.58,0,0.82}
\definecolor{backgroundColour}{rgb}{0.95,0.95,0.95}

\lstdefinestyle{CStyle}{
    backgroundcolor=\color{backgroundColour},   
    keywordstyle=\color{mGreen},
    numberstyle=\tiny\color{mGray},
    breakatwhitespace=false,         
    breaklines=true,                 
    captionpos=b,                    
    keepspaces=true,                 
    numbers=none,                    
    numbersep=5pt,                  
    showspaces=false,                
    showstringspaces=false,
    showtabs=false,                  
    tabsize=2,
    language=C,
    basicstyle=\footnotesize\ttfamily ,
    extendedchars=\true ,
}

% выбираем размер листа А4, все поля ставим по 3см
\geometry{a4paper,left=30mm,top=30mm,bottom=30mm,right=30mm}

\setcounter{secnumdepth}{0}      % отключаем нумерацию секций
\setcounter{tocdepth}{2}

\usepgfplotslibrary{fillbetween} % для изображения областей на графиках

\begin{document}
\begin{titlepage}
    \begin{center}
	{\small \sc Московский государственный университет \\имени М.~В.~Ломоносова\\
	Факультет вычислительной математики и кибернетики\\}
	\hrulefill
	\vfill
	{\large \bf Компьютерный практикум по учебному курсу}\\
	~\\
	{\Large \bf <<ВВЕДЕНИЕ В ЧИСЛЕННЫЕ МЕТОДЫ>>}\\ 
	~\\
	~\\
	~\\
	{\Large \bf ЗАДАНИЕ № 2}\\
	~\\
	{\bf Численные методы решения дифференциальных уравнений}\\
	~\\
	{\large \bf ОТЧЕТ}\\
	{\bf о выполненном задании}\\
	{студента 201 учебной группы факультета ВМК МГУ}\\
	{Галустова Артемия Львовича}
    \end{center}
    
    \begin{center}
	\vfill
	{\small гор. Москва\\2020 год}
    \end{center}
\end{titlepage}

\tableofcontents
\newpage
\section{Подвариант 1}
\subsection{Цель работы}
Освоить методы Рунге-Кутты второго и четвертого порядка точности,
применяемые для численного решения задачи Коши для дифференциального
уравнения (или системы дифференциальных уравнений) первого порядка.

\subsection{Постановка задачи}
Рассматривается обыкновенное дифференциальное уравнение первого порядка,
разрешенное относительно производной и имеющее вид:
\begin{align*}
\frac{dy}{dx}=f(x,y), x_0 < x,
\end{align*}

с дополнительным начальным условием, заданным в точке $x = x_0$ :
\begin{align*}
y(x_0)=y_0.
\end{align*}

Предполагается, что  функция $f = f (x, y)$ такова, что
гарантирует существование и единственность решения задачи Коши.
В том случае, если рассматривается не одно дифференциальное уравнение, а система обыкновенных дифференциальных уравнений первого порядка,
разрешенных
относительно
производных
неизвестных
функций, то
соответствующая задача Коши имеет вид (на примере двух дифференциальных
уравнений):
\begin{align*}
\begin{cases}
\frac{dx}{dt}=f_1(t,x,y),\\
\frac{dy}{dt}=f_2(t,x,y),
\end{cases}
\end{align*}
\par
Дополнительные (начальные) условия задаются в точке $x = x_0$:
\begin{align*}
x(t_0) = x_0, y(t_0)=y_0.
\end{align*}

Также предполагается, что $f_1, f_2$ заданы так, что это
гарантирует существование и единственность решения задачи Коши, но
уже для системы обыкновенных дифференциальных уравнений первого порядка в
форме, разрешенной относительно производных неизвестных функций.
\par
Требуется найти численное решение для данных задач Коши.

\subsection{Цели и задачи практической работы}
\begin{enumerate}
\item
Решить задачу Коши наиболее известными и широко
используемыми на практике методами Рунге-Кутта второго и четвертого
порядка
точности,
аппроксимировав
дифференциальную
задачу
соответствующей разностной схемой (на равномерной сетке); полученное
конечно-разностное уравнение (или уравнения в случае системы),
представляющее фактически некоторую рекуррентную формулу, просчитать
численно;
\item
Найти численное решение задачи и построить его график;
\item
Найденное численное решение сравнить с точным решением
дифференциального уравнения (подобрать специальные тесты, где
аналитические решения находятся в классе элементарных функций, для проверки и построения графиков использован онлайн сервис символьных вычислений Wolfaram One).

\end{enumerate}
\newpage
\subsection{Описание метода решения}
\subsubsection{Метод Рунге-Кутты для обыкновенного дифференциального уравнения первого порядка разрешённого относительно производной}
Пусть дано обыкновенное дифференциальное уравнение первого порядка разрешённое относительно производной на отрезке $[x_0, x_0 + l], l > 0$:
\begin{align*}
y'(x)=f(x,y(x)), x \in [x_0, x_0 + l]
\end{align*}
Рассмотрим сетку равномерную сетку $x_i = x_0 + ih, i = 0, 1, 2 ... n, h = \frac{l}{n}$. Где $n$ - число шагов и является параметром алгоритма. Метод Рунге-Кутты второго порядка точности задаёт рекуррентную формулу
\begin{align*}
&k_1 = f(x_i, y_i)\\
&k_2 = f(x_i + h, y_i + h * k_1)\\
&y_{i + 1} = y_i + \frac{h}{2}(k_1 + k_2).
\end{align*} 
\par
Метод четвёртого порядка точности задаёт рекуррентную формулу
\begin{align*}
y_{i+1} = \frac{h}{6} (k_1 +2k_2 + 2k_3 + k_4)
\end{align*}
где
\begin{align*}
&k_1 = f(x_i, y_i)\\
&k_2 = f(x_i + \frac{h}{2}, y_i + \frac{h}{2}k_1)\\
&k_3 = f(x_i + \frac{h}{2}, y_i + \frac{h}{2}k_2)\\
&k_4 = f(x_i + h, y_i + hk_3)
\end{align*}

\subsubsection{Метод Рунге-Кутты для системы двух обыкновенных дифференциальных уравнений первого порядка разрешённых относительно производной}
Пусть дана систем двух обыкновенных дифференциальных уравнений первого порядка разрешённых относительно производной на отрезке $[t_0, t_0 + l], l > 0$:
\begin{align*}
\begin{cases}
x'(t)=f_1(t,x,y)\\
y'(t)=f_2(t,x,y)
\end{cases}
\end{align*}
Рассмотрим сетку равномерную сетку $t_i = t_0 + ih, i = 0, 1, 2 ... n, h = \frac{l}{n}$. Где $n$ - число шагов и является параметром алгоритма. Метод Рунге-Кутты второго порядка точности для системы аналогичен таковому для одного уравнения и задаёт рекуррентную формулу
\begin{align*}
x_{i+1} = x_i + \frac{h}{2}(k_1 + k_2)\\
y_{i+1} = y_i + \frac{h}{2}(l_1 + l_2)
\end{align*}
где
\begin{align*}
&k_1 = f_1(t_i, x_i, y_i),~~ l_1 = f_2(t_i, x_i, y_i)\\
&k_2 = f_1(t_i + h, x_i + h k_1, y_i + h l_1),~~ l_2 = f_2(t_i + h, x_i + h k_1, y_i + h l_1)\\
\end{align*}
Метод четвёртого порядка точности задаёт рекуррентную формулу
\begin{align*}
x_{i+1} = x_i + \frac{h}{6}(k_1 + 2 k_2 + 2 k_3 + k_4)\\
y_{i+1} = y_i + \frac{h}{6}(l_1 + 2 l_2 + 2 l_3 + l_4)
\end{align*}
где
\begin{align*}
&k_1 = f_1(t_i, x_i, y_i), ~~ l_1 = f_2(t_i, x_i, y_i)\\
&k_2 = f_1(t_i + \frac{h}{2}, x_i + \frac{h}{2}k_1, y_i + \frac{h}{2}l_1), ~~ l_2 = f_2(t_i + \frac{h}{2}, x_i + \frac{h}{2}k_1, y_i + \frac{h}{2}l_1)\\
&k_3 = f_1(t_i + \frac{h}{2}, x_i + \frac{h}{2}k_2, y_i + \frac{h}{2}l_2), ~~ l_3 = f_2(t_i + \frac{h}{2}, x_i + \frac{h}{2}k_2, y_i + \frac{h}{2}l_2)\\
&k_4 = f_1(t_i + h, x_i + h k_3, y_i + h l_3), ~~ l_4 = f_2(t_i + h, x_i + h k_3, y_i + h l_3)
\end{align*}
\newpage
\subsection{Описание программы}
Данный алгоритм был реализован на языке С с применением стандартной библиотеки языка С. Ниже приведена часть файла RK.c содержащая реализацию описанных выше методов. Полный код программы доступен в разделе <<\nameref{source}>>, а также онлайн по адресу \url{https://github.com/NotLebedev/de_solver}.
\lstinputlisting[style=CStyle, firstline=5]{../RK.c}

\subsection{Тестирование}
Для проверки результатов выполнения программы на тестах был использован использован онлайн сервис символьных вычислений Wolfaram One. Были подобраны примеры с аналитическим решением, и построены графики их решений с наложением точек полученных в результате работы программы. Были проведены тесты для числа шагов $n = 10, 100, 1000$, а также для второго и четвёртого порядка точности.
\begin{enumerate}
\item
Уравнение $y'(x) = -\frac{y}{x}$ с начальными условиями $y(1)=3$. Аналитическое решение $y = \frac{3}{x}$.
\begin{figure}[h]
\centering
\includegraphics[width=0.7\textwidth]{test_1_1_2.png}
\caption{Второй порядок точности}
\centering
\includegraphics[width=0.7\textwidth]{test_1_1_4.png}
\caption{Четвёртый порядок точности}
\end{figure}
\par
Численное решения при любой точности метода и любом числе шагов близко к аналитическому.
\end{enumerate}
\end{document}